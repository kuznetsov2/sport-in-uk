\section{Автоматизированные методы распределения в коммерческой логистике}
\subsection{1}
История Английского Футбола. 

Игра футбол, несомненно, берет свое начало в Англии. Однако ранние записи о том, как человек пинает ногами что-то напоминающее мяч, восходят к китайской династии Хань 2000 лет назад. Есть записи о том, что среди древних греков и римлян, была распространена игра, известная как «Гарпастум», любопытная смесь футбола и регби. Во времена Юлия Цезаря римляне привезли ее в Англию. Именно в средние века игра завоевала популярность благодаря ежегодным матчам Масленицы, проводимым между соседними деревнями между командами с неограниченным числом игроков. В качестве мяча обычно использовался свиной мочевой пузырь, хотя в одном печально известном случае в Честере, во время празднования недавней победы над захватчиками использовалась голова мертвого викинга.

В конце 13-го и начале 14-го века уличные игры были настолько популярны в Лондоне, что торговцы призвали короля Эдуарда II запретить игру в городе.  Таким образом, 13 апреля 1314 года вступил в силу первый в истории футбольный запрет. Несмотря на угрозу тюремного заключения, запрет оказался довольно неэффективным, Короли Эдвард III, Ричард II, Генри II и Джеймс III также предпринимали попытки запретить футбол, но без какого-либо успеха. Единственный запрет, который хоть как-то работал, был наложен Оливером Кромвелем. Игры на короткое время исчезли, но вновь появились с еще большей популярностью после Реставрации Стюартов в 1660 году.

Популярность игры продолжала расти во времена Елизаветы, Автор Strutt в своей публикации «Спорт и развлечения» дал представление о футболе в 18 в. Он описал две команды с равным количеством игроков, которые выстроились между двумя воротами, сделанными из палок, расположенных на расстоянии около 1 ярда (0,91 м). Ворота были на расстоянии 80-100 ярдов (73-91 м.) друг от друга. Он также писал: «Мяч, который обычно сделан из надутого мочевого пузыря и обшит кожей, устанавливался посреди земли, и задача каждой команды состоит в том, чтобы забить его в ворота своих противников, что означает победу в игре."

Переживая различные попытки запрета, футбол оставался довольно дикой игрой. Какие-либо правила отсутствовали до середины 1840-х годов, когда желание команд, особенно в государственных школах, играть друг против друга, сделало отсутствие общепринятого свода правил проблемой, который нужно решить. Так, в 1848 году г-н Х. де Уинтон и г-н Дж. К. Тринг, два футболиста из Кембриджского университета, созвали встречу с представителями основных государственных школ, и на собрании, которое длилось 7 часов 55 минут, был выработан первый формальный свод правил игры в футбол.

Старейший в мире футбольный клуб, Шеффилд ФК, появился на свет в 1855 году. Самым старым из существующих клубов лиги является Ноттс Каунти, который был основан в 1862 году. Поскольку создавалось все больше и больше клубов, отсутствие руководящего органа стало вызывать проблемы. После встречи в октябре 1863 года в «Таверне масонов» на Грейт-Квин-стрит в Лондоне была образована «Футбольная ассоциация», в которой приняли участие г-н А. Пембер, первый президент и десять других членов-основателей.

Одним из главных приоритетов ФА было установить правила игры, которые должны соблюдать все члены. Первоначально были некоторые трудности, так как государственные школы все еще играли по своим собственным правилам, установленным на собрании 1848 года. В конце концов, они были убеждены и первые правила ФА были опубликованы в декабре 1862. Однако по-прежнему существовало серьезное недовольство правилами и многочисленные разногласия, в первую очередь, между сторонниками «Футбольной Ассоциации» и «Регби-юнион». Это привело к расколу и сторонники Регби-юнион покинули ФА для разработки своей собственной игры и создания «Футбольного союза регби».

К.В. Алкок.

В 1871 году тогдашний секретарь ФА г-н К. В. Алкок счел интересным проведение соревнований на вылет для команд всех стран, что ознаменовало рождение старейшего в мире футбольного соревнования «Кубок Футбольной Ассоциации».

По совпадению, г-н Алкок был первым, кто должен был представить первый в мире международный турнир в качестве капитана Уондерерс, выигравших первый в истории кубок Ассоциации, 16 марта 1872 на стадионе овал в Кеннингтоне обыграв в финале Ройал Энджинирс со счетом 1-0. ошибка В том же 1872 году Англия сыграла свой первый международный матч против Шотландии. Он был сыгран в Queen's Park в Шотландии и закончился вничью 0-0. В 1872 году также впервые был установлен размер мяча. Затем в 1875 году, лента, которая была натянута между двумя стойками ворот, была заменена на сплошную перекладину.

В марте 1873 года была сформирована шотландская FA для проведения игры к северу от границы, а в 1876 году была сформирована Уэльская FA для проведения игры в Уэльсе.

К 1880 году членство в ФА составило до 128 клубов и ассоциаций. В том числе 80 с юга Англии, 41 с севера, 6 из Шотландии и 1 из Австралии. 1882 год ознаменовался введением двуручного вбрасывания, а с 1885 года футбол стал считаться профессиональным видом деятельности.

К 1888 году игра остро нуждалась в реорганизации. Кубок Англии имел большой успех, однако еженедельные «дружеские» матчи часто отменялись в последнюю минуту из-за травм или транспортных затруднений. Болельщики часто прибывали на стадион чтобы узнать что матч отменен. Именно в это время Уильям МакГрегор, владелец магазина из Пертшира и игрок Астон Вилла, взял на себя инициативу и обратился к крупнейшим английским клубам того времени, предложив структурированную лигу, где каждая команда играет друг с другом дома и в гостях, в лиге в течение структурированного сезона, с двумя очками команде-победителю и по одному очку каждой команде в случае ничьей. Первоначальная встреча между клубами в Лондоне состоялась 22 марта 1888 года. Затем 17 апреля 1888 года на собрании двенадцати самых выдающихся клубов Англии в «Королевском отеле» в Манчестере была образована «Футбольная лига».



В нее вошли двенадцать клубов.

Акрингтон (Старые Красные)

Астон Вилла

Блэкберн Роверс

Болтон Уондерерс

Бернли

Дерби Каунти 

Эвертон

Ноттс Каунти 

Престон Норт Энд

Сток Сити

Вест Бромвич Альбион

Вулверхэмптон Уондерерс





ошибка

Футбольная лига стартовала 8 сентября 1888 года, когда Джек Гордон из Престон Норт Энд забил первый гол в истории. В тот первый сезон Престон Норт Энд (на фото справа) получил прозвище «Старые непобедимые», поскольку они выиграли первый чемпионат Футбольной лиги первого дивизиона, не проиграв ни одной игры.  В том же сезоне они также выиграли Кубок Англии, не пропустив ни одного гола.

В 1891 году Лига расширилась до четырнадцати клубов, судьи и арбитры были заменены рефери и двумя боковыми судьями, были впервые использованы сетки ворот и введены штрафы.

В 1892 году лига была снова расширена с добавлением нового дивизиона. Первоначальная лига называлась Дивизион 1 и была расширена до шестнадцати клубов. Был введен новый Второй Дивизион состоящий из двенадцати клубов, включая недавно отошедший Дарвен и одиннадцать новых клубов. Движение между двумя лигами было решено "Тестовыми матчами", сыгранными между тремя нижними клубами в Первом Дивизионе и тремя верхними клубами во Втором Дивизионе.

В 1893 году Второй Дивизион был расширен до пятнадцати клубов, и «Арсенал», «Ливерпуль», «Мидлсбро Айронополис» и «Ньюкасл Юнайтед» были впервые допущены в футбольную лигу.

В 1894 году Второй Дивизион был расширен до шестнадцати клубов. В течение этого периода некоторые клубы лиги продолжали играть в других, местных лигах и соревнованиях, таких как Лига объединённых округов, Палатинская лига и Кубок Берфорда.

В 1898 году оба дивизиона были расширены до восемнадцати клубов. Также в 1898 году было введено ежегодное повышение и понижение между дивизионами, в результате которого два нижних клуба первого дивизиона автоматически заменялись двумя лучшими командами второго дивизиона. В том же 1908 году был впервые разыгран «Суперкубок шерифа Лондона (Дьюара)», который позднее стал «Благотворительный кубок Футбольной ассоциации».

В 1900 году футбол впервые стал олимпийским видом спорта, Англия выступила в составе сборной Великобритании и завоевала золотую медаль.

В 1901 году «Ливерпуль» впервые выиграл чемпионат первого дивизиона, а титул они выиграли в общей сложности восемнадцать раз, что является рекордом лиги.

К началу 1900-х популярность игры распространилась по всему земному шару, особенно в европейских странах, которые познакомились с футболом благодаря английским школьникам еще в 1865 году. В начале 1904 года секретарь Нидерландской футбольной ассоциации Карл Антон Вильгельм Хиршманн обратился к ФА со своей идеей о создании международного руководящего органа для футбольной ассоциации. Он хотел, чтобы ФА, как самая старая футбольная ассоциация, играла ведущую роль. Тогдашний секретарь ФА согласился в принципе, но потребовалось немного времени, чтобы проконсультироваться с Исполнительным комитетом ФА, международным советом ФА и Ассоциациями Шотландии, Уэльса и Ирландии. Однако Роберт Герен, секретарь футбольного департамента Союза спортивной атлетики, не хотел ждать. Он написал в Футбольные ассоциации на континенте и попросил их внести вклад в формирование международного руководящего органа. После долгой переписки, 1 мая 1904 года Герен встретился с Луи Мюлингхаусом из Союза спортивных состязаний Бельгии. На этой встрече было решено, что Английская футбольная ассоциация при ее президенте лорде Киннаирде не будет участвовать в создании международной федерации. Поэтому Роберт Герен воспользовался возможностью и разослал приглашения учредительному собранию. Так, 21 мая 1904 года в штаб-квартире Союза спортивной атлетики Франции на улице Сен-Оноре, 229, Париж, был подписан учредительный акт и была создана «Международная федерация футбольного союза» или ФИФА. Было шесть стран-основателей: Франция, Бельгия, Дания, Нидерланды, Испания, Швеция и Швейцария.

В 1905 году «Ливерпуль» выиграл чемпионат второго дивизиона и был переведен в первый дивизион, где в следующем сезоне он выиграл чемпионат первого дивизиона, и стал единственным клубом, который выиграл второй, а затем первый дивизион в последовательных сезонах. Также в 1905 году ФА убедили вступить в ФИФА на том основании, что без самой старой и самой крупной ассоциации ФИФА было бы бессмысленным занятием. Также в начале сезона 1905-06 оба Первый и Второй дивизионы расширились до двадцати клубов.  В 1905 году также изменились правила: теперь вратарь должен оставаться на своей линии для пенальти.

В 1908 году Олимпиада проводилась в Лондоне, и снова Англия в составе сборной Великобритании завоевала золотую медаль.

В 1909 году «Суперкубок Англии», проводится впервые.

В 1912 году Англия снова в составе сборной Великобритании завоевала олимпийское золото.

В 1915 году «Первая мировая война» прервала обычную футбольную программу лиги. Хотя во время войны играли в футбол, из-за того, что большинство игроков проходили военную службу, игроки угадали ближайшую команду в свои казармы. В результате этого не были сопоставлены таблицы лиг по сезонам во время войны, футбольная лига возобновила нормальную программу лиги в 1919 году.

В 1919 году оба дивизиона были расширены до двадцати двух клубов. В том же 1919 году представители ФА приняли участие в конференции ФИФА в Париже, и из-за непримиримых разногласий с другими странами-членами по поводу продолжающегося присутствия недавних врагов времен Первой мировой войны начался бойкот ФИФА, который продолжался двадцать семь лет .

В 1920 футбольная лига была снова расширена, чтобы включить еще один дивизион. Новый Третий Дивизион включал еще двадцать два клуба

В 1921 году футбольная лига была расширена в последний раз до четырех дивизионов.  Первый дивизион, второй дивизион, третий дивизион (север) и третий дивизион (юг). Существующий Дивизион Три, плюс Округ Стокпорт (перенесенный из Дивизиона 2) и девятнадцать новых клубов, были разделены географически, чтобы сформировать эти две новые лиги с двадцатью клубами в Дивизион Три (Север) и двадцатью двумя клубами в Дивизион Три (Юг).

В 1923 году третий дивизион (север) был расширен до 22 клубов.

В 1924 году правила игры были изменены, чтобы сделать законным возможность забивать гол прямо из угла. Затем в 1925 году было изменено правило офсайда, теперь между нападающим и целью должно было оказаться не трое, а два игрока. 

28 мая 1928 года на конгрессе ФИФА в Амстердаме сбылась мечта тогдашнего президента ФИФА Жюля Римета (на фото слева), чтобы ФИФА провела свой собственный чемпионат мира. Предложение для Исполнительного комитета было принято, и конгресс согласился организовать «Кубок мира».

В 1930 году первый чемпионат мира состоялся в Уругвае, из-за продолжающегося бойкота Англии ФИФА, они не были приглашены принять участие в этом или двух последующих чемпионатах мира, проходивших в Италии и Франции. Широко распространено мнение о том, что у Англии был бы хороший шанс выиграть любой из этих чемпионатов мира, и этот бойкот отчасти является причиной относительно слабых результатов Англии на соревнованиях.

В 1939 году нумерация футболок игроков была впервые введена.

В 1946 году Жюль Римэ FA вынуждены прекратить их бойкот ФИФА.

В 1950 году Дивизион Три (Север) и Дивизион Три (Юг) были расширены до двадцати четырех клубов. Также в 1950 году Англия впервые приняла участие в чемпионате мира в Бразилии.  

В 1951 году правила были изменены, и белый футбол был введен.

В 1956 году «Кубок Европы» оспаривается впервые, и 22.02.1956 года в Ньюкасле состоялся первый прожженный матч между «Портсмутом» и «Ньюкасл Юнайтед», победа со счетом 2: 0.

В 1958 году третий дивизион (север) и третий дивизион (юг) были преобразованы в третий и четвертый дивизион с двадцатью четырьмя клубами в каждом. Лучшие двенадцать клубов из каждого Дивизиона вошли в Третий Дивизион, а нижние двенадцать --- в Четвертый. Также в 1958 году впервые проведен «Кубок УЕФА».

В 1960 году «Кубок Лиги» впервые появился в английском футболе. Это было детище Секретаря Футбольной лиги того времени, г-на Алана Хардакера, который рассматривал это как новый источник дохода для девяноста двух клубов лиги, которые могли участвовать. Астон Вилла выиграл первый финал в 1961 году.

В 1960 году впервые проводится «Межконтинентальный кубок».

В 1961 году  был впервые проведен «Кубок обладателей кубков», и отменена максимальная заработная плата для игроков.

В 1965 году впервые были разрешены замены, однако только для замены травмированных игроков.

В 1966 году правило о заменах было дополнительно изменено, чтобы разрешить их по любой причине.

30 июля 1966 года на стадионе Уэмбли Англия победила Западную Германию со счетом 4:2 в захватывающем финале Кубка мира. В конце этой игры было произнесено одно из самых известных футбольных высказываний всех времен, когда комментатор Кеннет Уолстенхолм сказал: Люди выходят на поле, они думают все закончилось. Теперь все.

В 1973 году была введена система повышения и понижения трех и трех ступеней.

В 1979 году была сформирована Футбольная конференция по футболу вне лиги из топ-команд Южной и Северной лиг (а затем и Лиги Истмиана), известная как Премьер-лига Альянса. Вопреки некоторым сообщениям, слово «альянс» просто используется в контексте союза между двумя лигами, не было никакой связи с Альянсом строительного общества. Первыми спонсорами были Gola, в 1984 году.

В 1981 году способ начисления очков в лиге был изменен, чтобы победитель получил больше преимуществ. Новая система дала три очка за победу и также одно очко в случае ничьей.

В 1982 году спонсорство футбольный начинается, когда кубок Лиги спонсирует молочный маркетинговый совет и становится чашкой молока.

В 1983 году спонсорских лиг вводится впервые. Дивизионы, один, два, три и четыре становятся Дивизионами Canon, один, два, три и четыре.

В 1986 году газета «Сегодня» приняла на себя спонсорство всех четырех дивизий, а Литлвудс перешла к кубку лиги. Также в этом году старый промоушен «Тестовые матчи» был вновь представлен в новом формате. Вместо того, чтобы нижние команды одного дивизиона играли против лучших команд другого, автоматическое понижение оставалось, и плей-офф был исключительно среди тех команд, которые искали повышение. Лучшие команды, две во втором дивизионе и три в третьем и четвертом дивизионе, по-прежнему автоматически продвигались, однако четыре команды, финишировавшие за ними, сыграли друг с другом за другое место повышения. Также были внесены изменения в Футбольную конференцию, началось автоматическое продвижение и вылет между Конференцией и Футбольной лигой. Несколько клубов лиги в последующие годы были спасены правилом, согласно которому сторону Конференции можно было продвигать только в том случае, если их возможности были в полном порядке.

В 1987 году Barclays перешла под спонсорство всех четырех подразделений. Также Дивизион Один был сокращен до двадцати одного клуба, в то время как Дивизион Два был расширен до двадцати трех клубов.

В 1988 году Дивизион Один был сокращен до двадцати клубов, а Дивизион Два расширился до двадцати четырех клубов.

В 1990 году правило офсайда было снова изменено, уровень атакующего с защитником больше не находится в офсайде. Также за профессиональный фол было совершено нарушение. И спонсорство кубка лиги перешло к Rumbelows.

В 1991 году Первый Дивизион был расширен до двадцати двух клубов, а Четвертый Дивизион сокращен до двадцати двух клубов.

В 1992 году была сформирована Премьер-лига Англии. Первый дивизион Barclays стал Премьер-лигой Англии, дивизоны Barclays, два, три и четыре стали дивизионами Barclays, один, два и три. Также спонсорство кубка лиги перешло Кока-коле.

В 1993 году Карлинг взял на себя спонсорство Премьер-лиги ФА, а Эндсли взял на себя спонсорство Дивизионов 1, 2 и 3.

В 1995 году премьерство ФК в Карлинге было сокращено до двадцати клубов, а в третьем дивизионе - до двадцати четырех.

В 1996 году «Спонсорство дивизионов 1, 2 и 3» перешло к Nationwide.

В 1998 году Спонсорство Кубка Лиги перешло к Уортингтону.

В 1999 году Кубок обладателей кубков Европы был проведен в последний раз. Финал был сыгран в Villa Park 19 мая 1999 года, его оспаривали «Реал Майорка» и возможные победители С.С. Лацио

В 2000 году  был впервые проведен «Чемпионат мира среди клубов».

В 2001 году Спонсорство Премьер-лиги перешло к Barclaycard.

В 2002 году между Футбольной лигой и Футбольной конференцией было введено два вниз два вверх повышения и понижения.

В 2003 году Спонсорство Кубка Лиги перешло к Карлинг.

В 2004 году Спонсорство Премьер-лиги перешло к Барклайз. Футбольная лига переименована в Первый Дивизион в Чемпионат, Второй Дивизион - в Первый Дивизион, а Третий Дивизион - во Второй Дивизион, в то же время спонсорство перешло к Coca Cola.

В 2004 году «Кубок Межконтинентального клуба» был проведен в последний раз. Со следующего года вместо него будет ежегодно проводится «Чемпионат мира по клубам». 