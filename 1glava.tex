\section{}
\subsection{}



Глобальное распространение видов спорта, которые имели свои истоки в Англии занимает центральное место в развитии современного спорта в 18 и 19 веках и является важным культурным наследием Британской империи. Современный футбол (футбол) общепризнанно возник в Англии. Футбольная ассоциация, первая футбольная организация, была основана в Англии в 1863 году, а первый футбольный матч между Англией и Шотландией — старейшее соперничество в спорте — прошел в Глазго в 1872 году. Английские футбольные фанаты могут болеть за три национальных дивизиона и знаменитую премьер-лигу, в которую входят такие легендарные клубы как Манчестер Юнайтед, Арсенал и Ливерпуль. В Шотландии есть три национальных дивизиона, а также своя Премьер-лига, в которой играют "Селтик" и "Рейнджерс" --- клубы из Глазго; Уэльс и Северная Ирландия также имеют национальную лигу. Команды из Шотландии и Англии регулярно выступают в международных соревнованиях. В 1966 году Англия проводила и выиграла Кубок мира, тем самым стала третьей принимающей страной, победившей в чемпионате.

Регби и крикет также давно пользуется большой популярностью в Великобритании. По легенде, регби появилось в 1823 году в школе города Регби в Англии. В 1871 был основан регбийный союз --- английский руководящая организация, а в 1895 году основана лига регби. Англия, Шотландия и Уэльс проводят клубные соревнования  и по регби-юнион и по регбилиг. Они также отправляют сборные команды для участия в чемпионате шести наций и Кубке мира. Происхождение крикета можно датировать 13-м веком в Англии, а соревнования между графствами были официально организованы в 19 веке. Международные матчи, известны как тестовые, начались в 1877 году противостоянием Англии и Австралии.

Великобритания участвовала во всех современных Олимпийских играх, начиная с первого соревнования в Афинах, в 1896 году. Великобритания принимала Игры три раза в Лондоне, в 1908, 1948 и 2012 годах. На соревнованиях 1896 года по тяжелой атлетике Лонсестон Эллиот стал первым британцем, завоевавшим золотую медаль, а в 1908 году фигуристка Мэдж Кейв Сайерс стала первой спортсменкой, завоевавшей медаль на зимних играх. Британские спортсмены завоевали сотни медалей за эти годы, демонстрируя особенно сильные результаты в легкой атлетике, теннисе, гребле, яхтинге и фигурном катании. Несколько британских спортсменов выступили с запоминающимися выступлениями на соревнованиях по легкой атлетике, в том числе спринтер Гарольд Абрахамс в 1920-х годах, бегуны на средние дистанции Себастьян Коу и Стив Оветт, а также двукратный золотой призер десятиборья Дейли Томпсон в 1970-х и 80-х годах. В 2000 году на летних Играх гребец Стив Редгрейв совершил редкий подвиг зарабатывая золотые медали на пяти играх подряд. На Играх 2012 года в Лондоне спортсмены, представляющие Великобританию, завоевали 65 медалей.

Великобритания является домом для нескольких важных международных спортивных соревнований. Открытый чемпионат, также известный за пределами Великобритании как British Open, - это турнир по гольфу, который проводится ежегодно, часто на всемирно известном поле в Сент-Эндрюсе в Шотландии. Чемпионат Англии (Уимблдона) - одно из ведущих соревнований по теннису в мире. Знаменитые соревнования по скачкам включают в себя Royal Ascot, Derby и Grand National steeplechase. Henley Royal Regatta - это первый мировой чемпионат по гребле.

Хотя климат Соединенного Королевства предрасполагает к пребыванию в помещении, британцы являются любителями активного отдыха на свежем воздухе. Это подтверждается наличием обширной сети пешеходных и велосипедных дорожек, национальных парков и других удобств. Особой популярностью пользуется Озерный край, в котором сохранилась живописная местность, о которой много говорят английские поэты; суровые Шотландское нагорье и острова Внутренние Гебриды; и горный валлийский регион национального парка Сноудония, притягивает альпинистов со всего мира. \cite{britannica}