\section{Теннис}

Теннис в не меньшей степени, чем футбол, считается настоящим английским видом спорта. Несмотря на то, что зародился он во Франции еще в XII веке, свой нынешний вид он приобрел именно в Англии, потому что французы играли в мячик рукой, а англичане, правда значительно позже, только в XVI веке, придумали ракетки.

Знаменитый теперь Уимблдонский турнир был основан в 1877 году. Тогда же и появилась традиция играть на травяных кортах. До сих пор этот чемпионат привлекает толпы туристов не только со всего мира, но и со всей Британии. И до сих пор здесь чтят традиции, которые зародились еще в позапрошлом веке.


 Уимблдонский теннисный турнир, как и многое в Англии, очень традиционное, можно даже сказать, консервативное мероприятие. На то есть причины, ведь Уимблдон – старейший международный теннисный турнир. Например, игрокам разрешается носить одежду только белого цвета. Многие пытались опротестовать это бесчеловечное правило, но организаторы турнира твердо стоят на своем. Также по старинке теннисистов здесь называют не иначе, как леди и джентльмены.

Еще одна старинная традиция касается мальчишек и девчонок, которые во время игры бегают за мячиками. Это так называемые ball boys и ball girls. Тысячи мальчишек и девчонок по всей стране мечтают попасть на стадион, однако отбор идет только в нескольких десятках местных школ. Каждую осень в этих школах начинается отбор претендентов, который длится несколько недель. Желающим испытать свои силы необходимо продемонстрировать ловкость и хорошую физическую подготовку. Прошедшие этот отбор 250 подростков отправляются на обучение непростому ремеслу, чтобы стать официальным помощником на Уимблдонском турнире. Занятия раз в неделю, но очень строгие. Те, кто прошел обучение, сравнивают его с курсами молодого бойца. От ball boys и ball girls требуется выправка и хорошая реакция. В качестве награды – почет и возможность попасть в кадр во время телевизионной трансляции теннисного матча на весь мир. Денег за эту работу не платят, только кормят бесплатно, но возможность провести три недели не в душном классе, а на самых известных в мире теннисных кортах в элегантной форменной одежде от Ralph Lauren, кажется, является куда лучшей мотивацией.