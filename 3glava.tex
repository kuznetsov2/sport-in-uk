\section{Крикет}

Крикет --- командный неконтактный вид спорта, входящий в семейство игр в которых используются бита и мяч. Происхождение игры начинается в 16 веке на юге Англии. К концу 18 столетия игра стала одним из национальных видов спорта. Крикет считался видом спорта для аристократов. Матчи игрались в белой форме, правила составляли увесистый том.

Данный вид спорта быстро распространился по миру, но не многие страны вошли в совет крикета. В основном игра существовала в странах (колониях) Британской империи.

Основными странами, входящими в ассоциацию настоящее время являются: Австралия, Англия, Бангладеш, Вест-Индия, Зимбабве, Индия, Новая Зеландия, Пакистан, Шри-Ланка и ЮАР.

Первый тестовый матч был проведён 15 марта 1877 года между сборными Англии и Австралии. Победу одержали Австралийцы к всеобщему удивлению. С тех пор, раз в полтора года эти сборные сражаются за победу.

В конце 2007 года Международное Общество Крикета уже не могло игнорировать нарастающий интерес к крикету, и придала ему статус признанного вида спорта, который дает право в будущем претендовать на включение в программу Олимпиады 2020 года.

Крикетный матч подразумевает соревнование двух команд, каждая из которых состоит из одиннадцати человек. Одни называются боулеры (падающие мяч), другие бэтсмены (отбивающие мяч битой). Также присутствует викет-кипер (игрок, находящийся сзади викета или калитки, его задача ловить мяч). Судья на поле ампаер.

Игра проходит на травяном поле, имеющем форму эллипса. В центре поля расположена прямоугольная земляная площадка --- питч. Длина питча составляет чуть более 20 метров, а ширина --- 3 метра. Игровые зоны на концах питча отделяются от его основного пространства специальными полосами (кризами).

Инвентарь для игры: крикетный мяч делается из пробки и покрывается кожей, он должен иметь массу от 155,9 до 163 граммов. Бывает двух цветов: красный для тест крикета и белый для однодневного крикета; бита имеет длину около 97 сантиметров и ширину --- 10,8 сантиметров, лопасть биты делают из дерева; деревянные калитки (от англ. wicket) расставлены на торцах питча и состоят из трех деревянных столбцов высотой 71 сантиметр.

Начало игры: чтобы узнать, кому отбивать, а кому кидать, капитаны команд бросают монетку. Выигравшая команда решает отбивать или кидать первой. Обычно капитаны выбирают первыми отбивать, поэтому оставляют на поле двух бэтсмены с каждой стороны питча. Капитан команды соперников выбирает двух боулеров и расставляет остальных игроков по полю. Команды по очереди отбивают мяч и играют в поле, пытаясь набрать максимальное количество очков или соответственно помешать в этом сопернику.

Правила игры: в рамках каждого хода (иннингса) подающий мяч боулер, играющей в поле команды, бросает мяч бэтсмену противника через всю длину питча. Каждый раз, когда бэтсмен отбил мяч, ему нужно успеть сделать ран ("пробежку) от своей калитки до другой, поэтому он пытается отразить бросок таким образом, чтобы мяч достиг границ поля или улетел достаточно далеко от противников. В случае успеха и при соблюдении некоторых других условий команда бэтсмена зарабатывает очки. Вместе с тем, выполнение некоторых игровых условий - поимка мяча соперником до касания земли, разрушение калитки бэтсмена и др. - выводит бэтсмена из игры. Иннингс продолжается до тех пор, пока десять бэтсменов отбивающей команды не будут выведены из игры, после чего команды меняются ролями. Боулеры сменяют друг друга по истечении шести подач, набор которых называется овером. Победителем становится команда, набравшая большее количество ранов (очков).

Хотя в основном, в крикете встречаются мужские команды, но всё же существует и женские. Первое упоминание о женском крикетном матче имело место в заметке в газете The Reading Mercury от 26 июля 1745 года.

В ней говорилось о матче, который состоялся между девушками из Хэмблдона и Брэмли. В 1887 году был создан первый женский крикетный клуб «Уайт Хэзар» \cite{cricet}.

У Англии нет собственной команды, вместо этого она выставляет совместную команду с Уэльсом. Сборная Англии по крикету, контролируется Советом Англии и Уэльса по крикету. Каждое лето две зарубежные сборные посещают и играют семь тестовых матчей и многочисленные турниры One Day Internationals, а зимой британцы отправляются в командировки за границу. Самый сильный противник команды --- австралийская команда, с которой она конкурирует за Пепел, один из самых известных трофеев в британском спорте.

Есть восемнадцать профессиональных клубов, 17 из них в Англии и один в Уэльсе. Каждое лето уездные клубы соревнуются в первоклассном чемпионате графства, который состоит из двух лиг из девяти команд и в которых матчи проводятся в течение четырех дней. Те же команды также играют в однодневной Национальной лиге, однодневных соревнованиях на выбывание под названием Friends Provident Trophy и короткометражном кубке Twenty20. Английские крикетные площадки включают в себя Lord's, Oval, Headingley, Old Trafford, Edgbaston и Trent Bridge. В последние годы территория Софийских садов Кардиффа становится все более популярной. Члены команды взяты из главных графств и включают в себя как английских, так и валлийских игроков. Это ни в коем случае не равнозначно футболу в финансах, посещаемости или освещении, но, тем не менее, имеет высокий авторитет. Вероятно, это второй самый широко освещаемый вид спорта в Англии и третий наиболее широко освещаемый вид спорта в Уэльсе, и за успехами сборной Англии следят многие люди, которые никогда не посещают живую игру.

У Шотландии и Ирландии есть свои команды по крикету, но игра не так популярна, как их команды не так успешны, как англичане и валлийцы. Ирландия не получала статус теста до 2017 года, а Шотландия до сих пор не имеет статуса теста. Поскольку Ирландия не проводила свои первые тесты до 2018 года, Шотландия до сих пор не проводит тесты, и только недавно они начали играть в полном формате One Day Internationals, многие шотландцы и ирландцы ранее играли в составе сборной Англии и Уэльса и капитанами; текущая сторона, например, включает Эоина Моргана, крикетиста из Дублина, который представлял Ирландию против Англии на чемпионате мира по крикету 2007 года, и капитан сборной Англии против Ирландии в 2011 году.

